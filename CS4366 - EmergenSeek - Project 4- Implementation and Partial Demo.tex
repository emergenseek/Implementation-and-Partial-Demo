\documentclass[10pt, a4paper]{article}
\usepackage[top=0.6in,bottom=1.0in,left=1.0in,right=1.0in]{geometry}
\usepackage{amsmath,amssymb}
\usepackage{hyperref}
\usepackage{graphicx,float,tikz}
\usepackage{listings}
\fontfamily{times}

\title{\large CS 4366: Senior Capstone Project \\ Dr. Sunho Lim \\ Project \#4 - Implementation and Partial Demo - Project Report \\ EmergenSeek}
\author{Suhas Bacchu \ Derek Fritz \ Kevon Manahan \ Annie Vo \ Simon Woldemichael}
\date{April 1, 2019}

\begin{document}

\maketitle
\vspace{-1cm}
\begin{abstract}
In this report, we describe and detail the progress that we have made so far in developing the backend and mobile client for the EmergenSeek application. The report will be broken up into sections such that, each section will answer the questions posed in the ``Deliverables'' section of the Project 4 statement file.
\end{abstract}

\section{Project Restatement} 
\label{sec:pr}
As a reminder, EmergenSeek is a mobile application which will provide users with multiuse, centralized emergency information and notification services. This application will provide friends and family members with priority connections in times of emergency or crisis. For the scope of the following two months, within this class, our main function requirements are as follows.
\begin{enumerate}
	\item[1.] S.O.S. button emergency broadcast --- The user shall be able to utilize the mobile client to press and hold an S.O.S. button for automated notification of contacts and emergency services.
	\item[2.] Emergency service locator --- The user shall be able to utilize the mobile client to search for emergency service (i.e. hospitals, pharmacies, police boxes)
	\item[3.] Periodic notifications to contacts (location-based polling) --- The user shall be able to utilize the mobile client to periodically send their location information to contacts.
	\item[4.] Granular permission definitions for contacts --- The user shall be given full control over what contacts receive what level of information.
	\item[5.] Lock screen display of health information for emergency services --- In the case of an S.O.S. situation, the user shall have their health information displayed for the convince of first responders.
\end{enumerate}

\section{Lambda Functions}
\label{sec:lf}
\par ~ This section should be used as a reference when there is any mention of an implemented Lambda function. The functions are prefixed with \texttt{ES} (short for EmergenSeek). This section will detail 6 things for each Lambda function:
	\begin{enumerate}
		\item[1.] The HTTP method necessary for interacting with the Lambda function.
		\item[2.] The API Gateway route associated with the function.
		\item[3.] The JavaScript Object Notation (JSON) \underline{request} body required by the function to produce desired output.
		\item[4.] The JSON \underline{response} returned by the API for a request.
		\item[5.] A simple-to-understand description of what the function performs as a result of the request.
		\item[6.] Services and API's that this function is dependent upon.
	\end{enumerate}
	
	\pagebreak
	\begin{enumerate}
	%% ESSendSMSNotification
	\item[a.] \textbf{ESSendSMSNotification}
		\begin{itemize}		
		\item[(i)] HTTP Method: POST
		\item[(ii)] API Gateway Route: /sms
		\item[(iii)] JSON Request Body:
			\begin{lstlisting}
{
	"user_id": String
	"type": int,
	"message": String
	"last_known_location": double[2]
}
			\end{lstlisting}
		\item[(iv)] JSON Response Body (Success):
			\begin{lstlisting}
{
	"body": String
}
			\end{lstlisting}
		\item[(v)] Description: First, this function will update the user's location in the database. Second, If the user provides 1 as an emergency type (SEVERE) this function will send an SMS message to the user's primary and secondary contacts and send a voice call to emergency services. If the user provides 2 as an emergency type (MILD) this function will only send an SMS message to the user's primary contacts. If the user provides 3 as an emergency type (CHECKIN) this function will only send an SMS message to the user's primary contacts with the contents of the "message" field. The "message" field is only required if "type" is 3. The response body of this function, on success, is simply a string stating that all operations completed successfully.
		\item[(vi)] Dependencies: DynamoDB, Twilio
		\end{itemize}
		
	%% ESSendEmergencyVoiceCall
	\item[b.] \textbf{ESSendEmergencyVoiceCall}
		\begin{itemize}		
		\item[(i)] HTTP Method: POST
		\item[(ii)] API Gateway Route: /voice
		\item[(iii)] JSON Request Body:
			\begin{lstlisting}
{
    "user_id": String,
    "last_known_location": double[2]
}
			\end{lstlisting}
		\item[(iv)] JSON Response Body (Success):
			\begin{lstlisting}
{
	"body": String
}
			\end{lstlisting}
		\item[(v)] Description: First, this function will update the user's location in the database. Second, the application will generate an XML document which conforms to Twilio's TwilML (Twilio Markup Language) specification \cite{two}. This document specifies how the programmable voice call is programmed. Third, this application will uplaod the XML document to Amazon's Simple Storage Service so that Twilio may access it for every phonecall that is dispatched as a result of a single request to this Lambda function. Lastly, this function will call the user's primary contacts. Currently, we have not integrated anything which will determine the correct 911 phone number depending on the user's location. The response body of this function, on success, is simply a string stating that all operations completed successfully.
		\item[(vi)] Dependencies: DynamoDB, Twilio, Simple Storage Service
		\end{itemize}
		
	%% ESPollLocation
	\item[c.] \textbf{ESPollLocation}
		\begin{itemize}		
		\item[(i)] HTTP Method: POST
		\item[(ii)] API Gateway Route: /poll
		\item[(iii)] JSON Request Body:
			\begin{lstlisting}
{
    "user_id": String,
    "last_known_location": double[2]
}
			\end{lstlisting}
		\item[(iv)] JSON Response Body (Success):
			\begin{lstlisting}
{
	"body": String
}
			\end{lstlisting}
		\item[(v)] Description: First, this function will update the user's location in the database. Next, this function will utilize MapQuest's Geocoding API to translate the provided latitude and longitude into a readable address. Lastly, this function will send this updated location, in readable address format, to the user's primary and secondary contacts. The decision to use MapQuest instead of Google's Geocoding API was made as a result of there being a very friendly Golang library to take care of the all of the client code (\url{https://github.com/jasonwinn/geocoder}). The response body of this function, on success, is simply a string stating that all operations completed successfully.
		\item[(vi)] Dependencies: DynamoDB, Twilio, MapQuest
		\end{itemize}
	
	%% ESGetLockScreenInfo
	\item[d.] \textbf{ESGetLockScreenInfo}
		\begin{itemize}		
		\item[(i)] HTTP Method: POST
		\item[(ii)] API Gateway Route: /lock
		\item[(iii)] JSON Request Body:
			\begin{lstlisting}
{
    "user_id": String,
}
			\end{lstlisting}
		\item[(iv)] JSON Response Body (Success):
			\begin{lstlisting}
{
    "first_name": String,
    "last_name":  String,
    "blood_type": String,
    "age": int,
    "primary_residence": {
      "line1": String,
      "line2": String,
      "city": String,
      "state": String,
      "country": String,
      "zip_code": String
    },
    "phone_pin": int,
    "email_address": String,
    "phone_number": String
}
			\end{lstlisting}
		\item[(v)] Description: First, this function will find a user with the matching user id in the database. Next the function will prepare a JSON object containing important user information for the user's lockscreen to be displayed on the client whenever the S.O.S. button is invoked. It is the responsibility of the client to invoke this function anytime the S.O.S. button is interacted with. It should be noted that while the functionality of this Lambda may also be accomplished through an HTTP GET request, using a POST request was simpler when defining the CloudFormation resource necessary for deploying this function from the CodePipeline of AWS CodeStar. More information on CodeStar may be found in the Continuous Integration / Continuous Deployment in section \ref{sec:cicd}.
		\item[(vi)] Dependencies: DynamoDB
		\end{itemize}
	
		%% ESServiceLocator
	\item[e.] \textbf{ESServiceLocator}
		\begin{itemize}		
		\item[(i)] HTTP Method: POST
		\item[(ii)] API Gateway Route: /locate
		\item[(iii)] JSON Request Body:
			\begin{lstlisting}
{
    "current_location": double[2]   
}
			\end{lstlisting}
		\item[(iv)] JSON Response Body (Success):
			\begin{lstlisting}
[
  {
    "location": {
      "lat": double,
      "lng": double
    },
    "name": String,
    "icon": String,
    "open": bool
  }
]
			\end{lstlisting}
		\item[(v)] Description: First, this function will query the Google Places API to retrieve pharmacies within a 20 mile radius of the latitude and longitude provided. Next, it will do the same for hospitals and merge the results of these two queries, after extracting only the latitude and longitude of each location of interest, the locations name, its open status, and an icon to identify the location on the client's map. 
		\item[(vi)] Dependencies: DynamoDB, Google Places API
		\end{itemize}
	\end{enumerate}

\subsection{Additional Functions}
\par ~ Since beginning implementation (specifically since creating our SRS document), we have found the need for four additional Lambda functions; ESCreateUser, ESGetUser, and ESUpdateOnLoad. These are necessary for creating new users of the application, retrieving existing users, and updating the user's location everytime the application is reopened, respectively. 

\section{Project Implementation}
\par ~ In this section we will describe how we implemented the project and give an overview of how the backend and mobile client are communicating, as well as the role of all of the technologies involved. 
\par ~ As a whole, and general rule when following a client-server architecture, the backend is an abstraction of all of the operations necessary for the mobile applications functionality. An end-user or a developer on the front-end team will never need to know of the serverless architectural pattern \cite{two} that our backend follows. Beginning from deriving out system features and requirements, we determined what Flutter components would be necessary to expose this functionality to the user and additionally determined the respective Golang AWS Lambda function to provide that functionality via an API call. That being said, the following subsections will describe precisely how this was achieved for each of the functionalities referenced in section \ref{sec:pr}. Reminder; details for any reference to a Lambda function prefixed with \texttt{ES} can be found in section \ref{sec:lf} 

\subsection{1 - S.O.S. button emergency broadcast} 
\par ~ This feature depends on three functionalities. On the client side, the user will press and hold an S.O.S. button. After a set amount of time, visualized via a loading bar, the user will have the option to select an emergency severity. At the moment the client only implements a ``severe'' and ``mild'' emergency option.
	\begin{itemize}
		\item[$\bullet$] If the user selects \texttt{severe}, the application will make a request to ESSendSMSNotification and ESSendEmergencyVoiceCall via two, asynchronous HTTP POST requests.
		\item[$\bullet$] If the user selects \texttt{mild}, the application will make a request to only ESSendSMSNotification via a single asynchronous HTTP POST request. 
		\item[$\bullet$] In the case of a \texttt{severe} request the application will make a request to ESGetLockScreenInfo to retrieve necessary personal information related to the user who pressed the S.O.S. button. 
	\end{itemize}


\section{Continuous Integration / Continuous Deployment}
\label{sec:cicd}
\par ~ In this section, we will d
% TODO: Add photo of CodeStar pipeline


\begin{thebibliography}{9}
\bibitem{one}
What is the AWS Serverless Application Model (AWS SAM)? ~ \url{https://docs.aws.amazon.com/serverless-application-model/latest/developerguide/what-is-sam.html}
\bibitem{two}
TwiML\textsuperscript{TM} for Programmable Voice ~ https://www.twilio.com/docs/voice/twiml

\end{thebibliography}
\end{document}

